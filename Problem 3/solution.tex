%%%%%%%%%%% - Preamble
\documentclass[11pt]{article}
\usepackage[sans, stdmargin, noindent]{rajeev}
\usepackage{lastpage}
\usepackage[inline]{asymptote}
\usepackage[normalem]{ulem}
\pagestyle{fancy}
\rhead{Last Updated: \today}
\lhead{$\RR^n$ Bonus Problems}

\begin{document}

\begin{center}
    \Large \textbf{$\RR^n$ Bonus Problem \#3}
\end{center}
\begin{center}
    \Large Rajeev Atla
\end{center}

\section{Problem}
\sout{Settlers of Catan}
A board game is played on a hexagonal grid of 19 tiles.
A 'traveler' token starts on the center tile.
Each turn a die is rolled to determine what neighboring tile the traveler moves to (all six directions equally likely).
The turn that the traveler leaves the board, the game ends.
What is the expected number of turns of the game?

\section{Diagram}
\begin{center}
    \begin{asy}
        size(10cm);

        pair[] coords = {(-3, -sqrt(3)), (-3, 0), (-3, sqrt(3)), 
            (-3/2, -3*sqrt(3)/2), (-3/2, -1*sqrt(3)/2), (-3/2, sqrt(3)/2), (-3/2, 3*sqrt(3)/2), 
            (0, -2*sqrt(3)), (0, -sqrt(3)), (0, 0), (0, sqrt(3)), (0, 2*sqrt(3)), 
            (3/2, -3*sqrt(3)/2), (3/2, -1*sqrt(3)/2), (3/2, sqrt(3)/2), (3/2, 3*sqrt(3)/2), 
            (3, -sqrt(3)), (3, 0), (3, sqrt(3))};




        for(int n = 0; n < coords.length; ++n){
            draw(shift(coords[n])*polygon(6), blue);
            dot(coords[n], red);
        }

        pair[] coords2 = {(-9/2, -3*sqrt(3)/2), (-9/2, -sqrt(3)/2), (-9/2, sqrt(3)/2), (-9/2, 3*sqrt(3)/2),
            (-3, -2*sqrt(3)), (-3, 2*sqrt(3)),
            (-3/2, -5*sqrt(3)/2), (-3/2, 5*sqrt(3)/2),
            (0, -3*sqrt(3)), (0, 3*sqrt(3)), 
            (3/2, -5*sqrt(3)/2), (3/2, 5*sqrt(3)/2), 
            (3, -2*sqrt(3)), (3, 2*sqrt(3)), 
            (9/2, -3*sqrt(3)/2), (9/2, -sqrt(3)/2), (9/2, sqrt(3)/2), (9/2, 3*sqrt(3)/2)};

       for(int n = 0; n < coords2.length; ++n){
           dot(coords2[n], green);
       }

       label("$A$", (0, 0), S);
       label("$B$", (0, sqrt(3)), S);
       label("$C_1$", (0, 2*sqrt(3)), S);
       label("$C_2$", (-3, sqrt(3)), S);
       label("$C_3$", (3, sqrt(3)), S);
       label("$C_4$", (-3, -sqrt(3)), S);
       label("$C_5$", (3, -sqrt(3)), S);
       label("$C_6$", (0, -2*sqrt(3)), S);
       label("$D_1$", (0, 3*sqrt(3)), S);
       label("$D_2$", (-3/2, 5*sqrt(3)/2), S);
       label("$D_2$", (3/2, 5*sqrt(3)/2), S);
       label("$E$", (-3/2, sqrt(3)/2), S);
    \end{asy}
\end{center}

\section{Solution}
We wish to find the expected value of the number of turns in the game, which we denote $N$.

\[
    \E{N} = \sum N\ \P{N}
\]

The dice is truly random, so there is no upper bound on $N$.
We therefore expect to evaluate an infinite series.

\subsection{$N=3$}
Observing that $N=3$ is the smallest possible number of turns, we explore it first.
Such a path could be $A \to B \to C_1 \to D_1$.
However, we also see that at $C_1$, there are 3 possible paths: $C_1 \to D_1$, $C_1 \to D_2$, and $C_1 \to D_3$.
Therefore, when the traveler is at $C_1$, there is a $\frac{1}{2}$ chance of the game ending on that turn.
In fact, this happens at all of $\{ C_1, C_2, C_3, C_4, C_5, C_6 \}$.
We have
\[
    \P{N=3} = \left ( \frac{1}{2} \right)\left (\frac{1}{6} \right) \left (\frac{1}{6} \right) \left (\frac{1}{2} \right) = \frac{1}{72}
\]
This is the only way to get $N=3$, so the contribution to the expected value is $3 \left ( \frac{1}{72} \right) = \frac{1}{24}$.

\subsection{General $N$}
We now generalize these results.
Since each move was made by a die roll, we see that a path with length $N$ that lies completely within the board has probability $\left (\frac{1}{6} \right)^N$.
Exiting the board is where things become more complicated.

There are two ways to exit the board: at corner or at non-corner (for lack of a better term).
There are 6 corners and 6 non-corners, so the probability of exiting through each is $\frac{1}{2}$ by symmetry.
At a corner $C_i$, there is a $\frac{3}{6} = \frac{1}{2}$ chance of the game ending in one more turn.
At a non-corner, there is $\frac{2}{6} = \frac{1}{3}$ chance of the game ending in one more turn.

Let the binary $C \in \{0, 1 \}$ be the condition of exiting through a corner ($1$) or a non-corner ($0$).
We then compute the expected value using the law of total expectation.

\begin{align*}
    \E{N} &= \sum \limits_{C \in \{0, 1 \}} \E{N \mid C} \P{C} \\ 
    &= \E{N \mid C = 0} \P{C=0} + \E{N \mid C = 1} \P{C = 1} \\
    &= \frac{1}{2} \E{N \mid C = 0} + \frac{1}{2} \E{N \mid C = 1} \\
    &= \frac{1}{2} \sum N\ \P{N \mid C = 0} + \frac{1}{2} \sum N\ \P{N \mid C = 1} \\
    &= 
\end{align*}


\end{document}
