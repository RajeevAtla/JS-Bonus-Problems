%%%%%%%%%%% - Preamble
\documentclass[11pt]{article}
\usepackage[sans, stdmargin, noindent]{rajeev}
\usepackage{lastpage}
\usepackage[inline]{asymptote}
\usepackage[normalem]{ulem}
\pagestyle{fancy}
\rhead{Last Updated: \today}
\lhead{$\RR^n$ Bonus Problems}

\newcommand\scalemath[2]{\scalebox{#1}{\mbox{\ensuremath{\displaystyle #2}}}} %https://tex.stackexchange.com/a/43067

\begin{document}

\begin{center}
    \Large \textbf{$\RR^n$ Bonus Problem \#3}
\end{center}
\begin{center}
    \Large Rajeev Atla
\end{center}

\section{Problem}
\sout{Settlers of Catan}
A board game is played on a hexagonal grid of 19 tiles.
A 'traveler' token starts on the center tile.
Each turn a die is rolled to determine what neighboring tile the traveler moves to (all six directions equally likely).
The turn that the traveler leaves the board, the game ends.
What is the expected number of turns of the game?

\section{Diagram}
\begin{center}
    \begin{asy}
        size(10cm);

        pair[] coords = 
        {
            (0, 0),
            (0, sqrt(3)), (-3/2, sqrt(3)/2), (-3/2, -1*sqrt(3)/2), 
            (0, -sqrt(3)), (3/2, -1*sqrt(3)/2), (3/2, sqrt(3)/2),
            (3/2, 3*sqrt(3)/2), (0, 2*sqrt(3)), (-3/2, 3*sqrt(3)/2), 
            (-3, sqrt(3)), (-3, 0), (-3, -sqrt(3)), 
            (-3/2, -3*sqrt(3)/2), (0, -2*sqrt(3)), (3/2, -3*sqrt(3)/2), 
            (3, -sqrt(3)), (3, 0), (3, sqrt(3))
        };
        

        for(int n = 0; n < coords.length; ++n){
            draw(shift(coords[n])*polygon(6), blue);
            dot(coords[n], red);
            label(string(n), (coords[n]), S);
        }

        pair[] coords2 = 
        {
            (3, 2*sqrt(3)), (3/2, 5*sqrt(3)/2), (0, 3*sqrt(3)),
            (-3/2, 5*sqrt(3)/2), (-3, 2*sqrt(3)), (-9/2, 3*sqrt(3)/2),
            (-9/2, sqrt(3)/2), (-9/2, -sqrt(3)/2), (-9/2, -3*sqrt(3)/2),
            (-3, -2*sqrt(3)), (-3/2, -5*sqrt(3)/2), (0, -3*sqrt(3)),  
            (3/2, -5*sqrt(3)/2), (3, -2*sqrt(3)), (9/2, -3*sqrt(3)/2), 
            (9/2, -sqrt(3)/2), (9/2, sqrt(3)/2), (9/2, 3*sqrt(3)/2)
        };

        for(int n = 0; n < coords2.length; ++n){
           dot(coords2[n], green);
           label(string(n+19), (coords2[n]), S);
       }

    \end{asy}
\end{center}

\section{Solution}
We wish to find the expected value of the number of turns in the game, which we denote $N$.

\[
    \E{N} = \sum N\ \P{N}
\]

The dice is truly random, so there is no upper bound on $N$.
We note that this game is really akin to a Markov chain, in that it doesn't matter what the past states are.


\subsection{Notation}
Let $X_i \in [0, 36]$ be the current state, or position of the traveler.
The traveler always starts at position $X_0=0$.
The final state must be $X_N \in [19, 36]$.

\subsection{Transition Matrix}
Now that we've defined some notation, we can write the transition matrix $P$
Because a 37x37 matrix is cumbersome, we combine the states [19, 36] into a 

$$
P = 
\left ( \scalemath{0.4} { \begin{array}{cccccccccccccccccccc}
	p_{ 0, 0 } = 0 & p_{ 0, 1 } = \frac{1}{6} & p_{ 0, 2 } = \frac{1}{6} & p_{ 0, 3 } = \frac{1}{6} & p_{ 0, 4 } = \frac{1}{6} & p_{ 0, 5 } = \frac{1}{6} & p_{ 0, 6 } = \frac{1}{6} & p_{ 0, 7 } = 0 & p_{ 0, 8 } = 0 & p_{ 0, 9 } = 0 & p_{ 0, 10 } = 0 & p_{ 0, 11 } = 0 & p_{ 0, 12 } = 0 & p_{ 0, 13 } = 0 & p_{ 0, 14 } = 0 & p_{ 0, 15 } = 0 & p_{ 0, 16 } = 0 & p_{ 0, 17 } = 0 & p_{ 0, 18 } = 0 & p_{ 0, 19 } = 0\\
	p_{ 1, 0 } = \frac{1}{6} & p_{ 1, 1 } = 0 & p_{ 1, 2 } = \frac{1}{6} & p_{ 1, 3 } = 0 & p_{ 1, 4 } = 0 & p_{ 1, 5 } = 0 & p_{ 1, 6 } = \frac{1}{6} & p_{ 1, 7 } = \frac{1}{6} & p_{ 1, 8 } = \frac{1}{6} & p_{ 1, 9 } = \frac{1}{6} & p_{ 1, 10 } = 0 & p_{ 1, 11 } = 0 & p_{ 1, 12 } = 0 & p_{ 1, 13 } = 0 & p_{ 1, 14 } = 0 & p_{ 1, 15 } = 0 & p_{ 1, 16 } = 0 & p_{ 1, 17 } = 0 & p_{ 1, 18 } = 0 & p_{ 1, 19 } = 0\\
	p_{ 2, 0 } = \frac{1}{6} & p_{ 2, 1 } = \frac{1}{6} & p_{ 2, 2 } = 0 & p_{ 2, 3 } = \frac{1}{6} & p_{ 2, 4 } = 0 & p_{ 2, 5 } = 0 & p_{ 2, 6 } = 0 & p_{ 2, 7 } = 0 & p_{ 2, 8 } = 0 & p_{ 2, 9 } = \frac{1}{6} & p_{ 2, 10 } = \frac{1}{6} & p_{ 2, 11 } = \frac{1}{6} & p_{ 2, 12 } = 0 & p_{ 2, 13 } = 0 & p_{ 2, 14 } = 0 & p_{ 2, 15 } = 0 & p_{ 2, 16 } = 0 & p_{ 2, 17 } = 0 & p_{ 2, 18 } = 0 & p_{ 2, 19 } = 0\\
	p_{ 3, 0 } = \frac{1}{6} & p_{ 3, 1 } = 0 & p_{ 3, 2 } = \frac{1}{6} & p_{ 3, 3 } = 0 & p_{ 3, 4 } = \frac{1}{6} & p_{ 3, 5 } = 0 & p_{ 3, 6 } = 0 & p_{ 3, 7 } = 0 & p_{ 3, 8 } = 0 & p_{ 3, 9 } = 0 & p_{ 3, 10 } = 0 & p_{ 3, 11 } = \frac{1}{6} & p_{ 3, 12 } = \frac{1}{6} & p_{ 3, 13 } = \frac{1}{6} & p_{ 3, 14 } = 0 & p_{ 3, 15 } = 0 & p_{ 3, 16 } = 0 & p_{ 3, 17 } = 0 & p_{ 3, 18 } = 0 & p_{ 3, 19 } = 0\\
	p_{ 4, 0 } = \frac{1}{6} & p_{ 4, 1 } = 0 & p_{ 4, 2 } = 0 & p_{ 4, 3 } = \frac{1}{6} & p_{ 4, 4 } = 0 & p_{ 4, 5 } = \frac{1}{6} & p_{ 4, 6 } = 0 & p_{ 4, 7 } = 0 & p_{ 4, 8 } = 0 & p_{ 4, 9 } = 0 & p_{ 4, 10 } = 0 & p_{ 4, 11 } = 0 & p_{ 4, 12 } = 0 & p_{ 4, 13 } = \frac{1}{6} & p_{ 4, 14 } = \frac{1}{6} & p_{ 4, 15 } = \frac{1}{6} & p_{ 4, 16 } = 0 & p_{ 4, 17 } = 0 & p_{ 4, 18 } = 0 & p_{ 4, 19 } = 0\\
	p_{ 5, 0 } = \frac{1}{6} & p_{ 5, 1 } = 0 & p_{ 5, 2 } = 0 & p_{ 5, 3 } = 0 & p_{ 5, 4 } = \frac{1}{6} & p_{ 5, 5 } = 0 & p_{ 5, 6 } = \frac{1}{6} & p_{ 5, 7 } = 0 & p_{ 5, 8 } = 0 & p_{ 5, 9 } = 0 & p_{ 5, 10 } = 0 & p_{ 5, 11 } = 0 & p_{ 5, 12 } = 0 & p_{ 5, 13 } = 0 & p_{ 5, 14 } = 0 & p_{ 5, 15 } = \frac{1}{6} & p_{ 5, 16 } = \frac{1}{6} & p_{ 5, 17 } = \frac{1}{6} & p_{ 5, 18 } = 0 & p_{ 5, 19 } = 0\\
	p_{ 6, 0 } = \frac{1}{6} & p_{ 6, 1 } = \frac{1}{6} & p_{ 6, 2 } = 0 & p_{ 6, 3 } = 0 & p_{ 6, 4 } = 0 & p_{ 6, 5 } = \frac{1}{6} & p_{ 6, 6 } = 0 & p_{ 6, 7 } = \frac{1}{6} & p_{ 6, 8 } = 0 & p_{ 6, 9 } = 0 & p_{ 6, 10 } = 0 & p_{ 6, 11 } = 0 & p_{ 6, 12 } = 0 & p_{ 6, 13 } = 0 & p_{ 6, 14 } = 0 & p_{ 6, 15 } = 0 & p_{ 6, 16 } = 0 & p_{ 6, 17 } = \frac{1}{6} & p_{ 6, 18 } = \frac{1}{6} & p_{ 6, 19 } = 0\\
	p_{ 7, 0 } = 0 & p_{ 7, 1 } = \frac{1}{6} & p_{ 7, 2 } = 0 & p_{ 7, 3 } = 0 & p_{ 7, 4 } = 0 & p_{ 7, 5 } = 0 & p_{ 7, 6 } = \frac{1}{6} & p_{ 7, 7 } = 0 & p_{ 7, 8 } = \frac{1}{6} & p_{ 7, 9 } = 0 & p_{ 7, 10 } = 0 & p_{ 7, 11 } = 0 & p_{ 7, 12 } = 0 & p_{ 7, 13 } = 0 & p_{ 7, 14 } = 0 & p_{ 7, 15 } = 0 & p_{ 7, 16 } = 0 & p_{ 7, 17 } = 0 & p_{ 7, 18 } = \frac{1}{6} & p_{ 7, 19 } = \frac{1}{3}\\
	p_{ 8, 0 } = 0 & p_{ 8, 1 } = \frac{1}{6} & p_{ 8, 2 } = 0 & p_{ 8, 3 } = 0 & p_{ 8, 4 } = 0 & p_{ 8, 5 } = 0 & p_{ 8, 6 } = 0 & p_{ 8, 7 } = \frac{1}{6} & p_{ 8, 8 } = 0 & p_{ 8, 9 } = \frac{1}{6} & p_{ 8, 10 } = 0 & p_{ 8, 11 } = 0 & p_{ 8, 12 } = 0 & p_{ 8, 13 } = 0 & p_{ 8, 14 } = 0 & p_{ 8, 15 } = 0 & p_{ 8, 16 } = 0 & p_{ 8, 17 } = 0 & p_{ 8, 18 } = 0 & p_{ 8, 19 } = \frac{1}{2}\\
	p_{ 9, 0 } = 0 & p_{ 9, 1 } = \frac{1}{6} & p_{ 9, 2 } = \frac{1}{6} & p_{ 9, 3 } = 0 & p_{ 9, 4 } = 0 & p_{ 9, 5 } = 0 & p_{ 9, 6 } = 0 & p_{ 9, 7 } = 0 & p_{ 9, 8 } = \frac{1}{6} & p_{ 9, 9 } = 0 & p_{ 9, 10 } = \frac{1}{6} & p_{ 9, 11 } = 0 & p_{ 9, 12 } = 0 & p_{ 9, 13 } = 0 & p_{ 9, 14 } = 0 & p_{ 9, 15 } = 0 & p_{ 9, 16 } = 0 & p_{ 9, 17 } = 0 & p_{ 9, 18 } = 0 & p_{ 9, 19 } = \frac{1}{3}\\
	p_{ 10, 0 } = 0 & p_{ 10, 1 } = 0 & p_{ 10, 2 } = \frac{1}{6} & p_{ 10, 3 } = 0 & p_{ 10, 4 } = 0 & p_{ 10, 5 } = 0 & p_{ 10, 6 } = 0 & p_{ 10, 7 } = 0 & p_{ 10, 8 } = 0 & p_{ 10, 9 } = \frac{1}{6} & p_{ 10, 10 } = 0 & p_{ 10, 11 } = \frac{1}{6} & p_{ 10, 12 } = 0 & p_{ 10, 13 } = 0 & p_{ 10, 14 } = 0 & p_{ 10, 15 } = 0 & p_{ 10, 16 } = 0 & p_{ 10, 17 } = 0 & p_{ 10, 18 } = 0 & p_{ 10, 19 } = \frac{1}{2}\\
	p_{ 11, 0 } = 0 & p_{ 11, 1 } = 0 & p_{ 11, 2 } = \frac{1}{6} & p_{ 11, 3 } = \frac{1}{6} & p_{ 11, 4 } = 0 & p_{ 11, 5 } = 0 & p_{ 11, 6 } = 0 & p_{ 11, 7 } = 0 & p_{ 11, 8 } = 0 & p_{ 11, 9 } = 0 & p_{ 11, 10 } = \frac{1}{6} & p_{ 11, 11 } = 0 & p_{ 11, 12 } = \frac{1}{6} & p_{ 11, 13 } = 0 & p_{ 11, 14 } = 0 & p_{ 11, 15 } = 0 & p_{ 11, 16 } = 0 & p_{ 11, 17 } = 0 & p_{ 11, 18 } = 0 & p_{ 11, 19 } = \frac{1}{3}\\
	p_{ 12, 0 } = 0 & p_{ 12, 1 } = 0 & p_{ 12, 2 } = 0 & p_{ 12, 3 } = \frac{1}{6} & p_{ 12, 4 } = 0 & p_{ 12, 5 } = 0 & p_{ 12, 6 } = 0 & p_{ 12, 7 } = 0 & p_{ 12, 8 } = 0 & p_{ 12, 9 } = 0 & p_{ 12, 10 } = 0 & p_{ 12, 11 } = \frac{1}{6} & p_{ 12, 12 } = 0 & p_{ 12, 13 } = \frac{1}{6} & p_{ 12, 14 } = 0 & p_{ 12, 15 } = 0 & p_{ 12, 16 } = 0 & p_{ 12, 17 } = 0 & p_{ 12, 18 } = 0 & p_{ 12, 19 } = \frac{1}{2}\\
	p_{ 13, 0 } = 0 & p_{ 13, 1 } = 0 & p_{ 13, 2 } = 0 & p_{ 13, 3 } = \frac{1}{6} & p_{ 13, 4 } = \frac{1}{6} & p_{ 13, 5 } = 0 & p_{ 13, 6 } = 0 & p_{ 13, 7 } = 0 & p_{ 13, 8 } = 0 & p_{ 13, 9 } = 0 & p_{ 13, 10 } = 0 & p_{ 13, 11 } = 0 & p_{ 13, 12 } = \frac{1}{6} & p_{ 13, 13 } = 0 & p_{ 13, 14 } = \frac{1}{6} & p_{ 13, 15 } = 0 & p_{ 13, 16 } = 0 & p_{ 13, 17 } = 0 & p_{ 13, 18 } = 0 & p_{ 13, 19 } = \frac{1}{3}\\
	p_{ 14, 0 } = 0 & p_{ 14, 1 } = 0 & p_{ 14, 2 } = 0 & p_{ 14, 3 } = 0 & p_{ 14, 4 } = \frac{1}{6} & p_{ 14, 5 } = 0 & p_{ 14, 6 } = 0 & p_{ 14, 7 } = 0 & p_{ 14, 8 } = 0 & p_{ 14, 9 } = 0 & p_{ 14, 10 } = 0 & p_{ 14, 11 } = 0 & p_{ 14, 12 } = 0 & p_{ 14, 13 } = \frac{1}{6} & p_{ 14, 14 } = 0 & p_{ 14, 15 } = \frac{1}{6} & p_{ 14, 16 } = 0 & p_{ 14, 17 } = 0 & p_{ 14, 18 } = 0 & p_{ 14, 19 } = \frac{1}{2}\\
	p_{ 15, 0 } = 0 & p_{ 15, 1 } = 0 & p_{ 15, 2 } = 0 & p_{ 15, 3 } = 0 & p_{ 15, 4 } = \frac{1}{6} & p_{ 15, 5 } = \frac{1}{6} & p_{ 15, 6 } = 0 & p_{ 15, 7 } = 0 & p_{ 15, 8 } = 0 & p_{ 15, 9 } = 0 & p_{ 15, 10 } = 0 & p_{ 15, 11 } = 0 & p_{ 15, 12 } = 0 & p_{ 15, 13 } = 0 & p_{ 15, 14 } = \frac{1}{6} & p_{ 15, 15 } = 0 & p_{ 15, 16 } = \frac{1}{6} & p_{ 15, 17 } = 0 & p_{ 15, 18 } = 0 & p_{ 15, 19 } = \frac{1}{3}\\
	p_{ 16, 0 } = 0 & p_{ 16, 1 } = 0 & p_{ 16, 2 } = 0 & p_{ 16, 3 } = 0 & p_{ 16, 4 } = 0 & p_{ 16, 5 } = \frac{1}{6} & p_{ 16, 6 } = 0 & p_{ 16, 7 } = 0 & p_{ 16, 8 } = 0 & p_{ 16, 9 } = 0 & p_{ 16, 10 } = 0 & p_{ 16, 11 } = 0 & p_{ 16, 12 } = 0 & p_{ 16, 13 } = 0 & p_{ 16, 14 } = 0 & p_{ 16, 15 } = \frac{1}{6} & p_{ 16, 16 } = 0 & p_{ 16, 17 } = \frac{1}{6} & p_{ 16, 18 } = 0 & p_{ 16, 19 } = \frac{1}{2}\\
	p_{ 17, 0 } = 0 & p_{ 17, 1 } = 0 & p_{ 17, 2 } = 0 & p_{ 17, 3 } = 0 & p_{ 17, 4 } = 0 & p_{ 17, 5 } = \frac{1}{6} & p_{ 17, 6 } = \frac{1}{6} & p_{ 17, 7 } = 0 & p_{ 17, 8 } = 0 & p_{ 17, 9 } = 0 & p_{ 17, 10 } = 0 & p_{ 17, 11 } = 0 & p_{ 17, 12 } = 0 & p_{ 17, 13 } = 0 & p_{ 17, 14 } = 0 & p_{ 17, 15 } = 0 & p_{ 17, 16 } = \frac{1}{6} & p_{ 17, 17 } = 0 & p_{ 17, 18 } = \frac{1}{6} & p_{ 17, 19 } = \frac{1}{3}\\
	p_{ 18, 0 } = 0 & p_{ 18, 1 } = 0 & p_{ 18, 2 } = 0 & p_{ 18, 3 } = 0 & p_{ 18, 4 } = 0 & p_{ 18, 5 } = 0 & p_{ 18, 6 } = \frac{1}{6} & p_{ 18, 7 } = \frac{1}{6} & p_{ 18, 8 } = 0 & p_{ 18, 9 } = 0 & p_{ 18, 10 } = 0 & p_{ 18, 11 } = 0 & p_{ 18, 12 } = 0 & p_{ 18, 13 } = 0 & p_{ 18, 14 } = 0 & p_{ 18, 15 } = 0 & p_{ 18, 16 } = 0 & p_{ 18, 17 } = \frac{1}{6} & p_{ 18, 18 } = 0 & p_{ 18, 19 } = \frac{1}{2}\\
	p_{ 19, 0 } = 0 & p_{ 19, 1 } = 0 & p_{ 19, 2 } = 0 & p_{ 19, 3 } = 0 & p_{ 19, 4 } = 0 & p_{ 19, 5 } = 0 & p_{ 19, 6 } = 0 & p_{ 19, 7 } = 0 & p_{ 19, 8 } = 0 & p_{ 19, 9 } = 0 & p_{ 19, 10 } = 0 & p_{ 19, 11 } = 0 & p_{ 19, 12 } = 0 & p_{ 19, 13 } = 0 & p_{ 19, 14 } = 0 & p_{ 19, 15 } = 0 & p_{ 19, 16 } = 0 & p_{ 19, 17 } = 0 & p_{ 19, 18 } = 0 & p_{ 19, 19 } = 1\\
\end{array} } \right)
$$


We also write the matrix $Q$, which doesn't have any absorbing states.

$$
Q = 
\left ( \scalemath{0.4} { \begin{array}{ccccccccccccccccccc}
	p_{ 0, 0 } = 0 & p_{ 0, 1 } = \frac{1}{6} & p_{ 0, 2 } = \frac{1}{6} & p_{ 0, 3 } = \frac{1}{6} & p_{ 0, 4 } = \frac{1}{6} & p_{ 0, 5 } = \frac{1}{6} & p_{ 0, 6 } = \frac{1}{6} & p_{ 0, 7 } = 0 & p_{ 0, 8 } = 0 & p_{ 0, 9 } = 0 & p_{ 0, 10 } = 0 & p_{ 0, 11 } = 0 & p_{ 0, 12 } = 0 & p_{ 0, 13 } = 0 & p_{ 0, 14 } = 0 & p_{ 0, 15 } = 0 & p_{ 0, 16 } = 0 & p_{ 0, 17 } = 0 & p_{ 0, 18 } = 0\\
	p_{ 1, 0 } = \frac{1}{6} & p_{ 1, 1 } = 0 & p_{ 1, 2 } = \frac{1}{6} & p_{ 1, 3 } = 0 & p_{ 1, 4 } = 0 & p_{ 1, 5 } = 0 & p_{ 1, 6 } = \frac{1}{6} & p_{ 1, 7 } = \frac{1}{6} & p_{ 1, 8 } = \frac{1}{6} & p_{ 1, 9 } = \frac{1}{6} & p_{ 1, 10 } = 0 & p_{ 1, 11 } = 0 & p_{ 1, 12 } = 0 & p_{ 1, 13 } = 0 & p_{ 1, 14 } = 0 & p_{ 1, 15 } = 0 & p_{ 1, 16 } = 0 & p_{ 1, 17 } = 0 & p_{ 1, 18 } = 0\\
	p_{ 2, 0 } = \frac{1}{6} & p_{ 2, 1 } = \frac{1}{6} & p_{ 2, 2 } = 0 & p_{ 2, 3 } = \frac{1}{6} & p_{ 2, 4 } = 0 & p_{ 2, 5 } = 0 & p_{ 2, 6 } = 0 & p_{ 2, 7 } = 0 & p_{ 2, 8 } = 0 & p_{ 2, 9 } = \frac{1}{6} & p_{ 2, 10 } = \frac{1}{6} & p_{ 2, 11 } = \frac{1}{6} & p_{ 2, 12 } = 0 & p_{ 2, 13 } = 0 & p_{ 2, 14 } = 0 & p_{ 2, 15 } = 0 & p_{ 2, 16 } = 0 & p_{ 2, 17 } = 0 & p_{ 2, 18 } = 0\\
	p_{ 3, 0 } = \frac{1}{6} & p_{ 3, 1 } = 0 & p_{ 3, 2 } = \frac{1}{6} & p_{ 3, 3 } = 0 & p_{ 3, 4 } = \frac{1}{6} & p_{ 3, 5 } = 0 & p_{ 3, 6 } = 0 & p_{ 3, 7 } = 0 & p_{ 3, 8 } = 0 & p_{ 3, 9 } = 0 & p_{ 3, 10 } = 0 & p_{ 3, 11 } = \frac{1}{6} & p_{ 3, 12 } = \frac{1}{6} & p_{ 3, 13 } = \frac{1}{6} & p_{ 3, 14 } = 0 & p_{ 3, 15 } = 0 & p_{ 3, 16 } = 0 & p_{ 3, 17 } = 0 & p_{ 3, 18 } = 0\\
	p_{ 4, 0 } = \frac{1}{6} & p_{ 4, 1 } = 0 & p_{ 4, 2 } = 0 & p_{ 4, 3 } = \frac{1}{6} & p_{ 4, 4 } = 0 & p_{ 4, 5 } = \frac{1}{6} & p_{ 4, 6 } = 0 & p_{ 4, 7 } = 0 & p_{ 4, 8 } = 0 & p_{ 4, 9 } = 0 & p_{ 4, 10 } = 0 & p_{ 4, 11 } = 0 & p_{ 4, 12 } = 0 & p_{ 4, 13 } = \frac{1}{6} & p_{ 4, 14 } = \frac{1}{6} & p_{ 4, 15 } = \frac{1}{6} & p_{ 4, 16 } = 0 & p_{ 4, 17 } = 0 & p_{ 4, 18 } = 0\\
	p_{ 5, 0 } = \frac{1}{6} & p_{ 5, 1 } = 0 & p_{ 5, 2 } = 0 & p_{ 5, 3 } = 0 & p_{ 5, 4 } = \frac{1}{6} & p_{ 5, 5 } = 0 & p_{ 5, 6 } = \frac{1}{6} & p_{ 5, 7 } = 0 & p_{ 5, 8 } = 0 & p_{ 5, 9 } = 0 & p_{ 5, 10 } = 0 & p_{ 5, 11 } = 0 & p_{ 5, 12 } = 0 & p_{ 5, 13 } = 0 & p_{ 5, 14 } = 0 & p_{ 5, 15 } = \frac{1}{6} & p_{ 5, 16 } = \frac{1}{6} & p_{ 5, 17 } = \frac{1}{6} & p_{ 5, 18 } = 0\\
	p_{ 6, 0 } = \frac{1}{6} & p_{ 6, 1 } = \frac{1}{6} & p_{ 6, 2 } = 0 & p_{ 6, 3 } = 0 & p_{ 6, 4 } = 0 & p_{ 6, 5 } = \frac{1}{6} & p_{ 6, 6 } = 0 & p_{ 6, 7 } = \frac{1}{6} & p_{ 6, 8 } = 0 & p_{ 6, 9 } = 0 & p_{ 6, 10 } = 0 & p_{ 6, 11 } = 0 & p_{ 6, 12 } = 0 & p_{ 6, 13 } = 0 & p_{ 6, 14 } = 0 & p_{ 6, 15 } = 0 & p_{ 6, 16 } = 0 & p_{ 6, 17 } = \frac{1}{6} & p_{ 6, 18 } = \frac{1}{6}\\
	p_{ 7, 0 } = 0 & p_{ 7, 1 } = \frac{1}{6} & p_{ 7, 2 } = 0 & p_{ 7, 3 } = 0 & p_{ 7, 4 } = 0 & p_{ 7, 5 } = 0 & p_{ 7, 6 } = \frac{1}{6} & p_{ 7, 7 } = 0 & p_{ 7, 8 } = \frac{1}{6} & p_{ 7, 9 } = 0 & p_{ 7, 10 } = 0 & p_{ 7, 11 } = 0 & p_{ 7, 12 } = 0 & p_{ 7, 13 } = 0 & p_{ 7, 14 } = 0 & p_{ 7, 15 } = 0 & p_{ 7, 16 } = 0 & p_{ 7, 17 } = 0 & p_{ 7, 18 } = \frac{1}{6}\\
	p_{ 8, 0 } = 0 & p_{ 8, 1 } = \frac{1}{6} & p_{ 8, 2 } = 0 & p_{ 8, 3 } = 0 & p_{ 8, 4 } = 0 & p_{ 8, 5 } = 0 & p_{ 8, 6 } = 0 & p_{ 8, 7 } = \frac{1}{6} & p_{ 8, 8 } = 0 & p_{ 8, 9 } = \frac{1}{6} & p_{ 8, 10 } = 0 & p_{ 8, 11 } = 0 & p_{ 8, 12 } = 0 & p_{ 8, 13 } = 0 & p_{ 8, 14 } = 0 & p_{ 8, 15 } = 0 & p_{ 8, 16 } = 0 & p_{ 8, 17 } = 0 & p_{ 8, 18 } = 0\\
	p_{ 9, 0 } = 0 & p_{ 9, 1 } = \frac{1}{6} & p_{ 9, 2 } = \frac{1}{6} & p_{ 9, 3 } = 0 & p_{ 9, 4 } = 0 & p_{ 9, 5 } = 0 & p_{ 9, 6 } = 0 & p_{ 9, 7 } = 0 & p_{ 9, 8 } = \frac{1}{6} & p_{ 9, 9 } = 0 & p_{ 9, 10 } = \frac{1}{6} & p_{ 9, 11 } = 0 & p_{ 9, 12 } = 0 & p_{ 9, 13 } = 0 & p_{ 9, 14 } = 0 & p_{ 9, 15 } = 0 & p_{ 9, 16 } = 0 & p_{ 9, 17 } = 0 & p_{ 9, 18 } = 0\\
	p_{ 10, 0 } = 0 & p_{ 10, 1 } = 0 & p_{ 10, 2 } = \frac{1}{6} & p_{ 10, 3 } = 0 & p_{ 10, 4 } = 0 & p_{ 10, 5 } = 0 & p_{ 10, 6 } = 0 & p_{ 10, 7 } = 0 & p_{ 10, 8 } = 0 & p_{ 10, 9 } = \frac{1}{6} & p_{ 10, 10 } = 0 & p_{ 10, 11 } = \frac{1}{6} & p_{ 10, 12 } = 0 & p_{ 10, 13 } = 0 & p_{ 10, 14 } = 0 & p_{ 10, 15 } = 0 & p_{ 10, 16 } = 0 & p_{ 10, 17 } = 0 & p_{ 10, 18 } = 0\\
	p_{ 11, 0 } = 0 & p_{ 11, 1 } = 0 & p_{ 11, 2 } = \frac{1}{6} & p_{ 11, 3 } = \frac{1}{6} & p_{ 11, 4 } = 0 & p_{ 11, 5 } = 0 & p_{ 11, 6 } = 0 & p_{ 11, 7 } = 0 & p_{ 11, 8 } = 0 & p_{ 11, 9 } = 0 & p_{ 11, 10 } = \frac{1}{6} & p_{ 11, 11 } = 0 & p_{ 11, 12 } = \frac{1}{6} & p_{ 11, 13 } = 0 & p_{ 11, 14 } = 0 & p_{ 11, 15 } = 0 & p_{ 11, 16 } = 0 & p_{ 11, 17 } = 0 & p_{ 11, 18 } = 0\\
	p_{ 12, 0 } = 0 & p_{ 12, 1 } = 0 & p_{ 12, 2 } = 0 & p_{ 12, 3 } = \frac{1}{6} & p_{ 12, 4 } = 0 & p_{ 12, 5 } = 0 & p_{ 12, 6 } = 0 & p_{ 12, 7 } = 0 & p_{ 12, 8 } = 0 & p_{ 12, 9 } = 0 & p_{ 12, 10 } = 0 & p_{ 12, 11 } = \frac{1}{6} & p_{ 12, 12 } = 0 & p_{ 12, 13 } = \frac{1}{6} & p_{ 12, 14 } = 0 & p_{ 12, 15 } = 0 & p_{ 12, 16 } = 0 & p_{ 12, 17 } = 0 & p_{ 12, 18 } = 0\\
	p_{ 13, 0 } = 0 & p_{ 13, 1 } = 0 & p_{ 13, 2 } = 0 & p_{ 13, 3 } = \frac{1}{6} & p_{ 13, 4 } = \frac{1}{6} & p_{ 13, 5 } = 0 & p_{ 13, 6 } = 0 & p_{ 13, 7 } = 0 & p_{ 13, 8 } = 0 & p_{ 13, 9 } = 0 & p_{ 13, 10 } = 0 & p_{ 13, 11 } = 0 & p_{ 13, 12 } = \frac{1}{6} & p_{ 13, 13 } = 0 & p_{ 13, 14 } = \frac{1}{6} & p_{ 13, 15 } = 0 & p_{ 13, 16 } = 0 & p_{ 13, 17 } = 0 & p_{ 13, 18 } = 0\\
	p_{ 14, 0 } = 0 & p_{ 14, 1 } = 0 & p_{ 14, 2 } = 0 & p_{ 14, 3 } = 0 & p_{ 14, 4 } = \frac{1}{6} & p_{ 14, 5 } = 0 & p_{ 14, 6 } = 0 & p_{ 14, 7 } = 0 & p_{ 14, 8 } = 0 & p_{ 14, 9 } = 0 & p_{ 14, 10 } = 0 & p_{ 14, 11 } = 0 & p_{ 14, 12 } = 0 & p_{ 14, 13 } = \frac{1}{6} & p_{ 14, 14 } = 0 & p_{ 14, 15 } = \frac{1}{6} & p_{ 14, 16 } = 0 & p_{ 14, 17 } = 0 & p_{ 14, 18 } = 0\\
	p_{ 15, 0 } = 0 & p_{ 15, 1 } = 0 & p_{ 15, 2 } = 0 & p_{ 15, 3 } = 0 & p_{ 15, 4 } = \frac{1}{6} & p_{ 15, 5 } = \frac{1}{6} & p_{ 15, 6 } = 0 & p_{ 15, 7 } = 0 & p_{ 15, 8 } = 0 & p_{ 15, 9 } = 0 & p_{ 15, 10 } = 0 & p_{ 15, 11 } = 0 & p_{ 15, 12 } = 0 & p_{ 15, 13 } = 0 & p_{ 15, 14 } = \frac{1}{6} & p_{ 15, 15 } = 0 & p_{ 15, 16 } = \frac{1}{6} & p_{ 15, 17 } = 0 & p_{ 15, 18 } = 0\\
	p_{ 16, 0 } = 0 & p_{ 16, 1 } = 0 & p_{ 16, 2 } = 0 & p_{ 16, 3 } = 0 & p_{ 16, 4 } = 0 & p_{ 16, 5 } = \frac{1}{6} & p_{ 16, 6 } = 0 & p_{ 16, 7 } = 0 & p_{ 16, 8 } = 0 & p_{ 16, 9 } = 0 & p_{ 16, 10 } = 0 & p_{ 16, 11 } = 0 & p_{ 16, 12 } = 0 & p_{ 16, 13 } = 0 & p_{ 16, 14 } = 0 & p_{ 16, 15 } = \frac{1}{6} & p_{ 16, 16 } = 0 & p_{ 16, 17 } = \frac{1}{6} & p_{ 16, 18 } = 0\\
	p_{ 17, 0 } = 0 & p_{ 17, 1 } = 0 & p_{ 17, 2 } = 0 & p_{ 17, 3 } = 0 & p_{ 17, 4 } = 0 & p_{ 17, 5 } = \frac{1}{6} & p_{ 17, 6 } = \frac{1}{6} & p_{ 17, 7 } = 0 & p_{ 17, 8 } = 0 & p_{ 17, 9 } = 0 & p_{ 17, 10 } = 0 & p_{ 17, 11 } = 0 & p_{ 17, 12 } = 0 & p_{ 17, 13 } = 0 & p_{ 17, 14 } = 0 & p_{ 17, 15 } = 0 & p_{ 17, 16 } = \frac{1}{6} & p_{ 17, 17 } = 0 & p_{ 17, 18 } = \frac{1}{6}\\
	p_{ 18, 0 } = 0 & p_{ 18, 1 } = 0 & p_{ 18, 2 } = 0 & p_{ 18, 3 } = 0 & p_{ 18, 4 } = 0 & p_{ 18, 5 } = 0 & p_{ 18, 6 } = \frac{1}{6} & p_{ 18, 7 } = \frac{1}{6} & p_{ 18, 8 } = 0 & p_{ 18, 9 } = 0 & p_{ 18, 10 } = 0 & p_{ 18, 11 } = 0 & p_{ 18, 12 } = 0 & p_{ 18, 13 } = 0 & p_{ 18, 14 } = 0 & p_{ 18, 15 } = 0 & p_{ 18, 16 } = 0 & p_{ 18, 17 } = \frac{1}{6} & p_{ 18, 18 } = 0\\
\end{array} } \right)
$$


$N = \pars{I - Q}^{-1}$ is known as the fundamental matrix of $P$.

$$
N^{-1} = 
\left ( \scalemath{0.35} { \begin{array}{cccccccccccccccccccc}
	p_{ 0, 0 } = 1 & p_{ 0, 1 } = \frac{1}{6} & p_{ 0, 2 } = \frac{1}{6} & p_{ 0, 3 } = \frac{1}{6} & p_{ 0, 4 } = \frac{1}{6} & p_{ 0, 5 } = \frac{1}{6} & p_{ 0, 6 } = \frac{1}{6} & p_{ 0, 7 } = 0 & p_{ 0, 8 } = 0 & p_{ 0, 9 } = 0 & p_{ 0, 10 } = 0 & p_{ 0, 11 } = 0 & p_{ 0, 12 } = 0 & p_{ 0, 13 } = 0 & p_{ 0, 14 } = 0 & p_{ 0, 15 } = 0 & p_{ 0, 16 } = 0 & p_{ 0, 17 } = 0 & p_{ 0, 18 } = 0 & p_{ 0, 19 } = 0\\
	p_{ 1, 0 } = \frac{1}{6} & p_{ 1, 1 } = 1 & p_{ 1, 2 } = \frac{1}{6} & p_{ 1, 3 } = 0 & p_{ 1, 4 } = 0 & p_{ 1, 5 } = 0 & p_{ 1, 6 } = \frac{1}{6} & p_{ 1, 7 } = \frac{1}{6} & p_{ 1, 8 } = \frac{1}{6} & p_{ 1, 9 } = \frac{1}{6} & p_{ 1, 10 } = 0 & p_{ 1, 11 } = 0 & p_{ 1, 12 } = 0 & p_{ 1, 13 } = 0 & p_{ 1, 14 } = 0 & p_{ 1, 15 } = 0 & p_{ 1, 16 } = 0 & p_{ 1, 17 } = 0 & p_{ 1, 18 } = 0 & p_{ 1, 19 } = 0\\
	p_{ 2, 0 } = \frac{1}{6} & p_{ 2, 1 } = \frac{1}{6} & p_{ 2, 2 } = 1 & p_{ 2, 3 } = \frac{1}{6} & p_{ 2, 4 } = 0 & p_{ 2, 5 } = 0 & p_{ 2, 6 } = 0 & p_{ 2, 7 } = 0 & p_{ 2, 8 } = 0 & p_{ 2, 9 } = \frac{1}{6} & p_{ 2, 10 } = \frac{1}{6} & p_{ 2, 11 } = \frac{1}{6} & p_{ 2, 12 } = 0 & p_{ 2, 13 } = 0 & p_{ 2, 14 } = 0 & p_{ 2, 15 } = 0 & p_{ 2, 16 } = 0 & p_{ 2, 17 } = 0 & p_{ 2, 18 } = 0 & p_{ 2, 19 } = 0\\
	p_{ 3, 0 } = \frac{1}{6} & p_{ 3, 1 } = 0 & p_{ 3, 2 } = \frac{1}{6} & p_{ 3, 3 } = 1 & p_{ 3, 4 } = \frac{1}{6} & p_{ 3, 5 } = 0 & p_{ 3, 6 } = 0 & p_{ 3, 7 } = 0 & p_{ 3, 8 } = 0 & p_{ 3, 9 } = 0 & p_{ 3, 10 } = 0 & p_{ 3, 11 } = \frac{1}{6} & p_{ 3, 12 } = \frac{1}{6} & p_{ 3, 13 } = \frac{1}{6} & p_{ 3, 14 } = 0 & p_{ 3, 15 } = 0 & p_{ 3, 16 } = 0 & p_{ 3, 17 } = 0 & p_{ 3, 18 } = 0 & p_{ 3, 19 } = 0\\
	p_{ 4, 0 } = \frac{1}{6} & p_{ 4, 1 } = 0 & p_{ 4, 2 } = 0 & p_{ 4, 3 } = \frac{1}{6} & p_{ 4, 4 } = 1 & p_{ 4, 5 } = \frac{1}{6} & p_{ 4, 6 } = 0 & p_{ 4, 7 } = 0 & p_{ 4, 8 } = 0 & p_{ 4, 9 } = 0 & p_{ 4, 10 } = 0 & p_{ 4, 11 } = 0 & p_{ 4, 12 } = 0 & p_{ 4, 13 } = \frac{1}{6} & p_{ 4, 14 } = \frac{1}{6} & p_{ 4, 15 } = \frac{1}{6} & p_{ 4, 16 } = 0 & p_{ 4, 17 } = 0 & p_{ 4, 18 } = 0 & p_{ 4, 19 } = 0\\
	p_{ 5, 0 } = \frac{1}{6} & p_{ 5, 1 } = 0 & p_{ 5, 2 } = 0 & p_{ 5, 3 } = 0 & p_{ 5, 4 } = \frac{1}{6} & p_{ 5, 5 } = 1 & p_{ 5, 6 } = \frac{1}{6} & p_{ 5, 7 } = 0 & p_{ 5, 8 } = 0 & p_{ 5, 9 } = 0 & p_{ 5, 10 } = 0 & p_{ 5, 11 } = 0 & p_{ 5, 12 } = 0 & p_{ 5, 13 } = 0 & p_{ 5, 14 } = 0 & p_{ 5, 15 } = \frac{1}{6} & p_{ 5, 16 } = \frac{1}{6} & p_{ 5, 17 } = \frac{1}{6} & p_{ 5, 18 } = 0 & p_{ 5, 19 } = 0\\
	p_{ 6, 0 } = \frac{1}{6} & p_{ 6, 1 } = \frac{1}{6} & p_{ 6, 2 } = 0 & p_{ 6, 3 } = 0 & p_{ 6, 4 } = 0 & p_{ 6, 5 } = \frac{1}{6} & p_{ 6, 6 } = 1 & p_{ 6, 7 } = \frac{1}{6} & p_{ 6, 8 } = 0 & p_{ 6, 9 } = 0 & p_{ 6, 10 } = 0 & p_{ 6, 11 } = 0 & p_{ 6, 12 } = 0 & p_{ 6, 13 } = 0 & p_{ 6, 14 } = 0 & p_{ 6, 15 } = 0 & p_{ 6, 16 } = 0 & p_{ 6, 17 } = \frac{1}{6} & p_{ 6, 18 } = \frac{1}{6} & p_{ 6, 19 } = 0\\
	p_{ 7, 0 } = 0 & p_{ 7, 1 } = \frac{1}{6} & p_{ 7, 2 } = 0 & p_{ 7, 3 } = 0 & p_{ 7, 4 } = 0 & p_{ 7, 5 } = 0 & p_{ 7, 6 } = \frac{1}{6} & p_{ 7, 7 } = 1 & p_{ 7, 8 } = \frac{1}{6} & p_{ 7, 9 } = 0 & p_{ 7, 10 } = 0 & p_{ 7, 11 } = 0 & p_{ 7, 12 } = 0 & p_{ 7, 13 } = 0 & p_{ 7, 14 } = 0 & p_{ 7, 15 } = 0 & p_{ 7, 16 } = 0 & p_{ 7, 17 } = 0 & p_{ 7, 18 } = \frac{1}{6} & p_{ 7, 19 } = \frac{1}{3}\\
	p_{ 8, 0 } = 0 & p_{ 8, 1 } = \frac{1}{6} & p_{ 8, 2 } = 0 & p_{ 8, 3 } = 0 & p_{ 8, 4 } = 0 & p_{ 8, 5 } = 0 & p_{ 8, 6 } = 0 & p_{ 8, 7 } = \frac{1}{6} & p_{ 8, 8 } = 1 & p_{ 8, 9 } = \frac{1}{6} & p_{ 8, 10 } = 0 & p_{ 8, 11 } = 0 & p_{ 8, 12 } = 0 & p_{ 8, 13 } = 0 & p_{ 8, 14 } = 0 & p_{ 8, 15 } = 0 & p_{ 8, 16 } = 0 & p_{ 8, 17 } = 0 & p_{ 8, 18 } = 0 & p_{ 8, 19 } = \frac{1}{2}\\
	p_{ 9, 0 } = 0 & p_{ 9, 1 } = \frac{1}{6} & p_{ 9, 2 } = \frac{1}{6} & p_{ 9, 3 } = 0 & p_{ 9, 4 } = 0 & p_{ 9, 5 } = 0 & p_{ 9, 6 } = 0 & p_{ 9, 7 } = 0 & p_{ 9, 8 } = \frac{1}{6} & p_{ 9, 9 } = 1 & p_{ 9, 10 } = \frac{1}{6} & p_{ 9, 11 } = 0 & p_{ 9, 12 } = 0 & p_{ 9, 13 } = 0 & p_{ 9, 14 } = 0 & p_{ 9, 15 } = 0 & p_{ 9, 16 } = 0 & p_{ 9, 17 } = 0 & p_{ 9, 18 } = 0 & p_{ 9, 19 } = \frac{1}{3}\\
	p_{ 10, 0 } = 0 & p_{ 10, 1 } = 0 & p_{ 10, 2 } = \frac{1}{6} & p_{ 10, 3 } = 0 & p_{ 10, 4 } = 0 & p_{ 10, 5 } = 0 & p_{ 10, 6 } = 0 & p_{ 10, 7 } = 0 & p_{ 10, 8 } = 0 & p_{ 10, 9 } = \frac{1}{6} & p_{ 10, 10 } = 1 & p_{ 10, 11 } = \frac{1}{6} & p_{ 10, 12 } = 0 & p_{ 10, 13 } = 0 & p_{ 10, 14 } = 0 & p_{ 10, 15 } = 0 & p_{ 10, 16 } = 0 & p_{ 10, 17 } = 0 & p_{ 10, 18 } = 0 & p_{ 10, 19 } = \frac{1}{2}\\
	p_{ 11, 0 } = 0 & p_{ 11, 1 } = 0 & p_{ 11, 2 } = \frac{1}{6} & p_{ 11, 3 } = \frac{1}{6} & p_{ 11, 4 } = 0 & p_{ 11, 5 } = 0 & p_{ 11, 6 } = 0 & p_{ 11, 7 } = 0 & p_{ 11, 8 } = 0 & p_{ 11, 9 } = 0 & p_{ 11, 10 } = \frac{1}{6} & p_{ 11, 11 } = 1 & p_{ 11, 12 } = \frac{1}{6} & p_{ 11, 13 } = 0 & p_{ 11, 14 } = 0 & p_{ 11, 15 } = 0 & p_{ 11, 16 } = 0 & p_{ 11, 17 } = 0 & p_{ 11, 18 } = 0 & p_{ 11, 19 } = \frac{1}{3}\\
	p_{ 12, 0 } = 0 & p_{ 12, 1 } = 0 & p_{ 12, 2 } = 0 & p_{ 12, 3 } = \frac{1}{6} & p_{ 12, 4 } = 0 & p_{ 12, 5 } = 0 & p_{ 12, 6 } = 0 & p_{ 12, 7 } = 0 & p_{ 12, 8 } = 0 & p_{ 12, 9 } = 0 & p_{ 12, 10 } = 0 & p_{ 12, 11 } = \frac{1}{6} & p_{ 12, 12 } = 1 & p_{ 12, 13 } = \frac{1}{6} & p_{ 12, 14 } = 0 & p_{ 12, 15 } = 0 & p_{ 12, 16 } = 0 & p_{ 12, 17 } = 0 & p_{ 12, 18 } = 0 & p_{ 12, 19 } = \frac{1}{2}\\
	p_{ 13, 0 } = 0 & p_{ 13, 1 } = 0 & p_{ 13, 2 } = 0 & p_{ 13, 3 } = \frac{1}{6} & p_{ 13, 4 } = \frac{1}{6} & p_{ 13, 5 } = 0 & p_{ 13, 6 } = 0 & p_{ 13, 7 } = 0 & p_{ 13, 8 } = 0 & p_{ 13, 9 } = 0 & p_{ 13, 10 } = 0 & p_{ 13, 11 } = 0 & p_{ 13, 12 } = \frac{1}{6} & p_{ 13, 13 } = 1 & p_{ 13, 14 } = \frac{1}{6} & p_{ 13, 15 } = 0 & p_{ 13, 16 } = 0 & p_{ 13, 17 } = 0 & p_{ 13, 18 } = 0 & p_{ 13, 19 } = \frac{1}{3}\\
	p_{ 14, 0 } = 0 & p_{ 14, 1 } = 0 & p_{ 14, 2 } = 0 & p_{ 14, 3 } = 0 & p_{ 14, 4 } = \frac{1}{6} & p_{ 14, 5 } = 0 & p_{ 14, 6 } = 0 & p_{ 14, 7 } = 0 & p_{ 14, 8 } = 0 & p_{ 14, 9 } = 0 & p_{ 14, 10 } = 0 & p_{ 14, 11 } = 0 & p_{ 14, 12 } = 0 & p_{ 14, 13 } = \frac{1}{6} & p_{ 14, 14 } = 1 & p_{ 14, 15 } = \frac{1}{6} & p_{ 14, 16 } = 0 & p_{ 14, 17 } = 0 & p_{ 14, 18 } = 0 & p_{ 14, 19 } = \frac{1}{2}\\
	p_{ 15, 0 } = 0 & p_{ 15, 1 } = 0 & p_{ 15, 2 } = 0 & p_{ 15, 3 } = 0 & p_{ 15, 4 } = \frac{1}{6} & p_{ 15, 5 } = \frac{1}{6} & p_{ 15, 6 } = 0 & p_{ 15, 7 } = 0 & p_{ 15, 8 } = 0 & p_{ 15, 9 } = 0 & p_{ 15, 10 } = 0 & p_{ 15, 11 } = 0 & p_{ 15, 12 } = 0 & p_{ 15, 13 } = 0 & p_{ 15, 14 } = \frac{1}{6} & p_{ 15, 15 } = 1 & p_{ 15, 16 } = \frac{1}{6} & p_{ 15, 17 } = 0 & p_{ 15, 18 } = 0 & p_{ 15, 19 } = \frac{1}{3}\\
	p_{ 16, 0 } = 0 & p_{ 16, 1 } = 0 & p_{ 16, 2 } = 0 & p_{ 16, 3 } = 0 & p_{ 16, 4 } = 0 & p_{ 16, 5 } = \frac{1}{6} & p_{ 16, 6 } = 0 & p_{ 16, 7 } = 0 & p_{ 16, 8 } = 0 & p_{ 16, 9 } = 0 & p_{ 16, 10 } = 0 & p_{ 16, 11 } = 0 & p_{ 16, 12 } = 0 & p_{ 16, 13 } = 0 & p_{ 16, 14 } = 0 & p_{ 16, 15 } = \frac{1}{6} & p_{ 16, 16 } = 1 & p_{ 16, 17 } = \frac{1}{6} & p_{ 16, 18 } = 0 & p_{ 16, 19 } = \frac{1}{2}\\
	p_{ 17, 0 } = 0 & p_{ 17, 1 } = 0 & p_{ 17, 2 } = 0 & p_{ 17, 3 } = 0 & p_{ 17, 4 } = 0 & p_{ 17, 5 } = \frac{1}{6} & p_{ 17, 6 } = \frac{1}{6} & p_{ 17, 7 } = 0 & p_{ 17, 8 } = 0 & p_{ 17, 9 } = 0 & p_{ 17, 10 } = 0 & p_{ 17, 11 } = 0 & p_{ 17, 12 } = 0 & p_{ 17, 13 } = 0 & p_{ 17, 14 } = 0 & p_{ 17, 15 } = 0 & p_{ 17, 16 } = \frac{1}{6} & p_{ 17, 17 } = 1 & p_{ 17, 18 } = \frac{1}{6} & p_{ 17, 19 } = \frac{1}{3}\\
	p_{ 18, 0 } = 0 & p_{ 18, 1 } = 0 & p_{ 18, 2 } = 0 & p_{ 18, 3 } = 0 & p_{ 18, 4 } = 0 & p_{ 18, 5 } = 0 & p_{ 18, 6 } = \frac{1}{6} & p_{ 18, 7 } = \frac{1}{6} & p_{ 18, 8 } = 0 & p_{ 18, 9 } = 0 & p_{ 18, 10 } = 0 & p_{ 18, 11 } = 0 & p_{ 18, 12 } = 0 & p_{ 18, 13 } = 0 & p_{ 18, 14 } = 0 & p_{ 18, 15 } = 0 & p_{ 18, 16 } = 0 & p_{ 18, 17 } = \frac{1}{6} & p_{ 18, 18 } = 1 & p_{ 18, 19 } = \frac{1}{2}\\
	p_{ 19, 0 } = 0 & p_{ 19, 1 } = 0 & p_{ 19, 2 } = 0 & p_{ 19, 3 } = 0 & p_{ 19, 4 } = 0 & p_{ 19, 5 } = 0 & p_{ 19, 6 } = 0 & p_{ 19, 7 } = 0 & p_{ 19, 8 } = 0 & p_{ 19, 9 } = 0 & p_{ 19, 10 } = 0 & p_{ 19, 11 } = 0 & p_{ 19, 12 } = 0 & p_{ 19, 13 } = 0 & p_{ 19, 14 } = 0 & p_{ 19, 15 } = 0 & p_{ 19, 16 } = 0 & p_{ 19, 17 } = 0 & p_{ 19, 18 } = 0 & p_{ 19, 19 } = 0\\
\end{array} } \right)
$$


$$
N = 
\left ( \scalemath{0.3} { \begin{array}{ccccccccccccccccccc}
	p_{ 0, 0 } = \frac{45}{29} & p_{ 0, 1 } = \frac{16}{29} & p_{ 0, 2 } = \frac{16}{29} & p_{ 0, 3 } = \frac{16}{29} & p_{ 0, 4 } = \frac{16}{29} & p_{ 0, 5 } = \frac{16}{29} & p_{ 0, 6 } = \frac{16}{29} & p_{ 0, 7 } = \frac{7}{29} & p_{ 0, 8 } = \frac{5}{29} & p_{ 0, 9 } = \frac{7}{29} & p_{ 0, 10 } = \frac{5}{29} & p_{ 0, 11 } = \frac{7}{29} & p_{ 0, 12 } = \frac{5}{29} & p_{ 0, 13 } = \frac{7}{29} & p_{ 0, 14 } = \frac{5}{29} & p_{ 0, 15 } = \frac{7}{29} & p_{ 0, 16 } = \frac{5}{29} & p_{ 0, 17 } = \frac{7}{29} & p_{ 0, 18 } = \frac{5}{29}\\
	p_{ 1, 0 } = \frac{16}{29} & p_{ 1, 1 } = \frac{992666}{674685} & p_{ 1, 2 } = \frac{348808}{674685} & p_{ 1, 3 } = \frac{191222}{674685} & p_{ 1, 4 } = \frac{160714}{674685} & p_{ 1, 5 } = \frac{191222}{674685} & p_{ 1, 6 } = \frac{348808}{674685} & p_{ 1, 7 } = \frac{6133}{14355} & p_{ 1, 8 } = \frac{87176}{224895} & p_{ 1, 9 } = \frac{6133}{14355} & p_{ 1, 10 } = \frac{14056}{74965} & p_{ 1, 11 } = \frac{173}{957} & p_{ 1, 12 } = \frac{21752}{224895} & p_{ 1, 13 } = \frac{1667}{14355} & p_{ 1, 14 } = \frac{5878}{74965} & p_{ 1, 15 } = \frac{1667}{14355} & p_{ 1, 16 } = \frac{21752}{224895} & p_{ 1, 17 } = \frac{173}{957} & p_{ 1, 18 } = \frac{14056}{74965}\\
	p_{ 2, 0 } = \frac{16}{29} & p_{ 2, 1 } = \frac{348808}{674685} & p_{ 2, 2 } = \frac{992666}{674685} & p_{ 2, 3 } = \frac{348808}{674685} & p_{ 2, 4 } = \frac{191222}{674685} & p_{ 2, 5 } = \frac{160714}{674685} & p_{ 2, 6 } = \frac{191222}{674685} & p_{ 2, 7 } = \frac{173}{957} & p_{ 2, 8 } = \frac{14056}{74965} & p_{ 2, 9 } = \frac{6133}{14355} & p_{ 2, 10 } = \frac{87176}{224895} & p_{ 2, 11 } = \frac{6133}{14355} & p_{ 2, 12 } = \frac{14056}{74965} & p_{ 2, 13 } = \frac{173}{957} & p_{ 2, 14 } = \frac{21752}{224895} & p_{ 2, 15 } = \frac{1667}{14355} & p_{ 2, 16 } = \frac{5878}{74965} & p_{ 2, 17 } = \frac{1667}{14355} & p_{ 2, 18 } = \frac{21752}{224895}\\
	p_{ 3, 0 } = \frac{16}{29} & p_{ 3, 1 } = \frac{191222}{674685} & p_{ 3, 2 } = \frac{348808}{674685} & p_{ 3, 3 } = \frac{992666}{674685} & p_{ 3, 4 } = \frac{348808}{674685} & p_{ 3, 5 } = \frac{191222}{674685} & p_{ 3, 6 } = \frac{160714}{674685} & p_{ 3, 7 } = \frac{1667}{14355} & p_{ 3, 8 } = \frac{21752}{224895} & p_{ 3, 9 } = \frac{173}{957} & p_{ 3, 10 } = \frac{14056}{74965} & p_{ 3, 11 } = \frac{6133}{14355} & p_{ 3, 12 } = \frac{87176}{224895} & p_{ 3, 13 } = \frac{6133}{14355} & p_{ 3, 14 } = \frac{14056}{74965} & p_{ 3, 15 } = \frac{173}{957} & p_{ 3, 16 } = \frac{21752}{224895} & p_{ 3, 17 } = \frac{1667}{14355} & p_{ 3, 18 } = \frac{5878}{74965}\\
	p_{ 4, 0 } = \frac{16}{29} & p_{ 4, 1 } = \frac{160714}{674685} & p_{ 4, 2 } = \frac{191222}{674685} & p_{ 4, 3 } = \frac{348808}{674685} & p_{ 4, 4 } = \frac{992666}{674685} & p_{ 4, 5 } = \frac{348808}{674685} & p_{ 4, 6 } = \frac{191222}{674685} & p_{ 4, 7 } = \frac{1667}{14355} & p_{ 4, 8 } = \frac{5878}{74965} & p_{ 4, 9 } = \frac{1667}{14355} & p_{ 4, 10 } = \frac{21752}{224895} & p_{ 4, 11 } = \frac{173}{957} & p_{ 4, 12 } = \frac{14056}{74965} & p_{ 4, 13 } = \frac{6133}{14355} & p_{ 4, 14 } = \frac{87176}{224895} & p_{ 4, 15 } = \frac{6133}{14355} & p_{ 4, 16 } = \frac{14056}{74965} & p_{ 4, 17 } = \frac{173}{957} & p_{ 4, 18 } = \frac{21752}{224895}\\
	p_{ 5, 0 } = \frac{16}{29} & p_{ 5, 1 } = \frac{191222}{674685} & p_{ 5, 2 } = \frac{160714}{674685} & p_{ 5, 3 } = \frac{191222}{674685} & p_{ 5, 4 } = \frac{348808}{674685} & p_{ 5, 5 } = \frac{992666}{674685} & p_{ 5, 6 } = \frac{348808}{674685} & p_{ 5, 7 } = \frac{173}{957} & p_{ 5, 8 } = \frac{21752}{224895} & p_{ 5, 9 } = \frac{1667}{14355} & p_{ 5, 10 } = \frac{5878}{74965} & p_{ 5, 11 } = \frac{1667}{14355} & p_{ 5, 12 } = \frac{21752}{224895} & p_{ 5, 13 } = \frac{173}{957} & p_{ 5, 14 } = \frac{14056}{74965} & p_{ 5, 15 } = \frac{6133}{14355} & p_{ 5, 16 } = \frac{87176}{224895} & p_{ 5, 17 } = \frac{6133}{14355} & p_{ 5, 18 } = \frac{14056}{74965}\\
	p_{ 6, 0 } = \frac{16}{29} & p_{ 6, 1 } = \frac{348808}{674685} & p_{ 6, 2 } = \frac{191222}{674685} & p_{ 6, 3 } = \frac{160714}{674685} & p_{ 6, 4 } = \frac{191222}{674685} & p_{ 6, 5 } = \frac{348808}{674685} & p_{ 6, 6 } = \frac{992666}{674685} & p_{ 6, 7 } = \frac{6133}{14355} & p_{ 6, 8 } = \frac{14056}{74965} & p_{ 6, 9 } = \frac{173}{957} & p_{ 6, 10 } = \frac{21752}{224895} & p_{ 6, 11 } = \frac{1667}{14355} & p_{ 6, 12 } = \frac{5878}{74965} & p_{ 6, 13 } = \frac{1667}{14355} & p_{ 6, 14 } = \frac{21752}{224895} & p_{ 6, 15 } = \frac{173}{957} & p_{ 6, 16 } = \frac{14056}{74965} & p_{ 6, 17 } = \frac{6133}{14355} & p_{ 6, 18 } = \frac{87176}{224895}\\
	p_{ 7, 0 } = \frac{7}{29} & p_{ 7, 1 } = \frac{6133}{14355} & p_{ 7, 2 } = \frac{173}{957} & p_{ 7, 3 } = \frac{1667}{14355} & p_{ 7, 4 } = \frac{1667}{14355} & p_{ 7, 5 } = \frac{173}{957} & p_{ 7, 6 } = \frac{6133}{14355} & p_{ 7, 7 } = \frac{375124}{301455} & p_{ 7, 8 } = \frac{10246}{33495} & p_{ 7, 9 } = \frac{49367}{301455} & p_{ 7, 10 } = \frac{460}{6699} & p_{ 7, 11 } = \frac{20338}{301455} & p_{ 7, 12 } = \frac{438}{11165} & p_{ 7, 13 } = \frac{15611}{301455} & p_{ 7, 14 } = \frac{438}{11165} & p_{ 7, 15 } = \frac{20338}{301455} & p_{ 7, 16 } = \frac{460}{6699} & p_{ 7, 17 } = \frac{49367}{301455} & p_{ 7, 18 } = \frac{10246}{33495}\\
	p_{ 8, 0 } = \frac{5}{29} & p_{ 8, 1 } = \frac{87176}{224895} & p_{ 8, 2 } = \frac{14056}{74965} & p_{ 8, 3 } = \frac{21752}{224895} & p_{ 8, 4 } = \frac{5878}{74965} & p_{ 8, 5 } = \frac{21752}{224895} & p_{ 8, 6 } = \frac{14056}{74965} & p_{ 8, 7 } = \frac{10246}{33495} & p_{ 8, 8 } = \frac{1836491}{1574265} & p_{ 8, 9 } = \frac{10246}{33495} & p_{ 8, 10 } = \frac{147473}{1574265} & p_{ 8, 11 } = \frac{460}{6699} & p_{ 8, 12 } = \frac{53687}{1574265} & p_{ 8, 13 } = \frac{438}{11165} & p_{ 8, 14 } = \frac{41159}{1574265} & p_{ 8, 15 } = \frac{438}{11165} & p_{ 8, 16 } = \frac{53687}{1574265} & p_{ 8, 17 } = \frac{460}{6699} & p_{ 8, 18 } = \frac{147473}{1574265}\\
	p_{ 9, 0 } = \frac{7}{29} & p_{ 9, 1 } = \frac{6133}{14355} & p_{ 9, 2 } = \frac{6133}{14355} & p_{ 9, 3 } = \frac{173}{957} & p_{ 9, 4 } = \frac{1667}{14355} & p_{ 9, 5 } = \frac{1667}{14355} & p_{ 9, 6 } = \frac{173}{957} & p_{ 9, 7 } = \frac{49367}{301455} & p_{ 9, 8 } = \frac{10246}{33495} & p_{ 9, 9 } = \frac{375124}{301455} & p_{ 9, 10 } = \frac{10246}{33495} & p_{ 9, 11 } = \frac{49367}{301455} & p_{ 9, 12 } = \frac{460}{6699} & p_{ 9, 13 } = \frac{20338}{301455} & p_{ 9, 14 } = \frac{438}{11165} & p_{ 9, 15 } = \frac{15611}{301455} & p_{ 9, 16 } = \frac{438}{11165} & p_{ 9, 17 } = \frac{20338}{301455} & p_{ 9, 18 } = \frac{460}{6699}\\
	p_{ 10, 0 } = \frac{5}{29} & p_{ 10, 1 } = \frac{14056}{74965} & p_{ 10, 2 } = \frac{87176}{224895} & p_{ 10, 3 } = \frac{14056}{74965} & p_{ 10, 4 } = \frac{21752}{224895} & p_{ 10, 5 } = \frac{5878}{74965} & p_{ 10, 6 } = \frac{21752}{224895} & p_{ 10, 7 } = \frac{460}{6699} & p_{ 10, 8 } = \frac{147473}{1574265} & p_{ 10, 9 } = \frac{10246}{33495} & p_{ 10, 10 } = \frac{1836491}{1574265} & p_{ 10, 11 } = \frac{10246}{33495} & p_{ 10, 12 } = \frac{147473}{1574265} & p_{ 10, 13 } = \frac{460}{6699} & p_{ 10, 14 } = \frac{53687}{1574265} & p_{ 10, 15 } = \frac{438}{11165} & p_{ 10, 16 } = \frac{41159}{1574265} & p_{ 10, 17 } = \frac{438}{11165} & p_{ 10, 18 } = \frac{53687}{1574265}\\
	p_{ 11, 0 } = \frac{7}{29} & p_{ 11, 1 } = \frac{173}{957} & p_{ 11, 2 } = \frac{6133}{14355} & p_{ 11, 3 } = \frac{6133}{14355} & p_{ 11, 4 } = \frac{173}{957} & p_{ 11, 5 } = \frac{1667}{14355} & p_{ 11, 6 } = \frac{1667}{14355} & p_{ 11, 7 } = \frac{20338}{301455} & p_{ 11, 8 } = \frac{460}{6699} & p_{ 11, 9 } = \frac{49367}{301455} & p_{ 11, 10 } = \frac{10246}{33495} & p_{ 11, 11 } = \frac{375124}{301455} & p_{ 11, 12 } = \frac{10246}{33495} & p_{ 11, 13 } = \frac{49367}{301455} & p_{ 11, 14 } = \frac{460}{6699} & p_{ 11, 15 } = \frac{20338}{301455} & p_{ 11, 16 } = \frac{438}{11165} & p_{ 11, 17 } = \frac{15611}{301455} & p_{ 11, 18 } = \frac{438}{11165}\\
	p_{ 12, 0 } = \frac{5}{29} & p_{ 12, 1 } = \frac{21752}{224895} & p_{ 12, 2 } = \frac{14056}{74965} & p_{ 12, 3 } = \frac{87176}{224895} & p_{ 12, 4 } = \frac{14056}{74965} & p_{ 12, 5 } = \frac{21752}{224895} & p_{ 12, 6 } = \frac{5878}{74965} & p_{ 12, 7 } = \frac{438}{11165} & p_{ 12, 8 } = \frac{53687}{1574265} & p_{ 12, 9 } = \frac{460}{6699} & p_{ 12, 10 } = \frac{147473}{1574265} & p_{ 12, 11 } = \frac{10246}{33495} & p_{ 12, 12 } = \frac{1836491}{1574265} & p_{ 12, 13 } = \frac{10246}{33495} & p_{ 12, 14 } = \frac{147473}{1574265} & p_{ 12, 15 } = \frac{460}{6699} & p_{ 12, 16 } = \frac{53687}{1574265} & p_{ 12, 17 } = \frac{438}{11165} & p_{ 12, 18 } = \frac{41159}{1574265}\\
	p_{ 13, 0 } = \frac{7}{29} & p_{ 13, 1 } = \frac{1667}{14355} & p_{ 13, 2 } = \frac{173}{957} & p_{ 13, 3 } = \frac{6133}{14355} & p_{ 13, 4 } = \frac{6133}{14355} & p_{ 13, 5 } = \frac{173}{957} & p_{ 13, 6 } = \frac{1667}{14355} & p_{ 13, 7 } = \frac{15611}{301455} & p_{ 13, 8 } = \frac{438}{11165} & p_{ 13, 9 } = \frac{20338}{301455} & p_{ 13, 10 } = \frac{460}{6699} & p_{ 13, 11 } = \frac{49367}{301455} & p_{ 13, 12 } = \frac{10246}{33495} & p_{ 13, 13 } = \frac{375124}{301455} & p_{ 13, 14 } = \frac{10246}{33495} & p_{ 13, 15 } = \frac{49367}{301455} & p_{ 13, 16 } = \frac{460}{6699} & p_{ 13, 17 } = \frac{20338}{301455} & p_{ 13, 18 } = \frac{438}{11165}\\
	p_{ 14, 0 } = \frac{5}{29} & p_{ 14, 1 } = \frac{5878}{74965} & p_{ 14, 2 } = \frac{21752}{224895} & p_{ 14, 3 } = \frac{14056}{74965} & p_{ 14, 4 } = \frac{87176}{224895} & p_{ 14, 5 } = \frac{14056}{74965} & p_{ 14, 6 } = \frac{21752}{224895} & p_{ 14, 7 } = \frac{438}{11165} & p_{ 14, 8 } = \frac{41159}{1574265} & p_{ 14, 9 } = \frac{438}{11165} & p_{ 14, 10 } = \frac{53687}{1574265} & p_{ 14, 11 } = \frac{460}{6699} & p_{ 14, 12 } = \frac{147473}{1574265} & p_{ 14, 13 } = \frac{10246}{33495} & p_{ 14, 14 } = \frac{1836491}{1574265} & p_{ 14, 15 } = \frac{10246}{33495} & p_{ 14, 16 } = \frac{147473}{1574265} & p_{ 14, 17 } = \frac{460}{6699} & p_{ 14, 18 } = \frac{53687}{1574265}\\
	p_{ 15, 0 } = \frac{7}{29} & p_{ 15, 1 } = \frac{1667}{14355} & p_{ 15, 2 } = \frac{1667}{14355} & p_{ 15, 3 } = \frac{173}{957} & p_{ 15, 4 } = \frac{6133}{14355} & p_{ 15, 5 } = \frac{6133}{14355} & p_{ 15, 6 } = \frac{173}{957} & p_{ 15, 7 } = \frac{20338}{301455} & p_{ 15, 8 } = \frac{438}{11165} & p_{ 15, 9 } = \frac{15611}{301455} & p_{ 15, 10 } = \frac{438}{11165} & p_{ 15, 11 } = \frac{20338}{301455} & p_{ 15, 12 } = \frac{460}{6699} & p_{ 15, 13 } = \frac{49367}{301455} & p_{ 15, 14 } = \frac{10246}{33495} & p_{ 15, 15 } = \frac{375124}{301455} & p_{ 15, 16 } = \frac{10246}{33495} & p_{ 15, 17 } = \frac{49367}{301455} & p_{ 15, 18 } = \frac{460}{6699}\\
	p_{ 16, 0 } = \frac{5}{29} & p_{ 16, 1 } = \frac{21752}{224895} & p_{ 16, 2 } = \frac{5878}{74965} & p_{ 16, 3 } = \frac{21752}{224895} & p_{ 16, 4 } = \frac{14056}{74965} & p_{ 16, 5 } = \frac{87176}{224895} & p_{ 16, 6 } = \frac{14056}{74965} & p_{ 16, 7 } = \frac{460}{6699} & p_{ 16, 8 } = \frac{53687}{1574265} & p_{ 16, 9 } = \frac{438}{11165} & p_{ 16, 10 } = \frac{41159}{1574265} & p_{ 16, 11 } = \frac{438}{11165} & p_{ 16, 12 } = \frac{53687}{1574265} & p_{ 16, 13 } = \frac{460}{6699} & p_{ 16, 14 } = \frac{147473}{1574265} & p_{ 16, 15 } = \frac{10246}{33495} & p_{ 16, 16 } = \frac{1836491}{1574265} & p_{ 16, 17 } = \frac{10246}{33495} & p_{ 16, 18 } = \frac{147473}{1574265}\\
	p_{ 17, 0 } = \frac{7}{29} & p_{ 17, 1 } = \frac{173}{957} & p_{ 17, 2 } = \frac{1667}{14355} & p_{ 17, 3 } = \frac{1667}{14355} & p_{ 17, 4 } = \frac{173}{957} & p_{ 17, 5 } = \frac{6133}{14355} & p_{ 17, 6 } = \frac{6133}{14355} & p_{ 17, 7 } = \frac{49367}{301455} & p_{ 17, 8 } = \frac{460}{6699} & p_{ 17, 9 } = \frac{20338}{301455} & p_{ 17, 10 } = \frac{438}{11165} & p_{ 17, 11 } = \frac{15611}{301455} & p_{ 17, 12 } = \frac{438}{11165} & p_{ 17, 13 } = \frac{20338}{301455} & p_{ 17, 14 } = \frac{460}{6699} & p_{ 17, 15 } = \frac{49367}{301455} & p_{ 17, 16 } = \frac{10246}{33495} & p_{ 17, 17 } = \frac{375124}{301455} & p_{ 17, 18 } = \frac{10246}{33495}\\
	p_{ 18, 0 } = \frac{5}{29} & p_{ 18, 1 } = \frac{14056}{74965} & p_{ 18, 2 } = \frac{21752}{224895} & p_{ 18, 3 } = \frac{5878}{74965} & p_{ 18, 4 } = \frac{21752}{224895} & p_{ 18, 5 } = \frac{14056}{74965} & p_{ 18, 6 } = \frac{87176}{224895} & p_{ 18, 7 } = \frac{10246}{33495} & p_{ 18, 8 } = \frac{147473}{1574265} & p_{ 18, 9 } = \frac{460}{6699} & p_{ 18, 10 } = \frac{53687}{1574265} & p_{ 18, 11 } = \frac{438}{11165} & p_{ 18, 12 } = \frac{41159}{1574265} & p_{ 18, 13 } = \frac{438}{11165} & p_{ 18, 14 } = \frac{53687}{1574265} & p_{ 18, 15 } = \frac{460}{6699} & p_{ 18, 16 } = \frac{147473}{1574265} & p_{ 18, 17 } = \frac{10246}{33495} & p_{ 18, 18 } = \frac{1836491}{1574265}\\
\end{array} } \right)
$$


In order to get the expected number of steps, we find $t_0$, where

$$
\bm{t} = N \bm{1}
$$

Here, $\bm{1}$ is a vector whose entries are all 1.

$$
\bm{t} = 
\left ( \scalemath{0.6} { \begin{array}{c}
\frac{213}{29}\\
\frac{184}{29}\\
\frac{184}{29}\\
\frac{184}{29}\\
\frac{184}{29}\\
\frac{184}{29}\\
\frac{184}{29}\\
\frac{124}{29}\\
\frac{101}{29}\\
\frac{124}{29}\\
\frac{101}{29}\\
\frac{124}{29}\\
\frac{101}{29}\\
\frac{124}{29}\\
\frac{101}{29}\\
\frac{124}{29}\\
\frac{101}{29}\\
\frac{124}{29}\\
\frac{101}{29}\\
\end{array} } \right)
$$


Finally, we see that $t_0 = \boxed{\frac{213}{29} \approx 7.345}$

\end{document}
