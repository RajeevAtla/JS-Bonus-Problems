%%%%%%%%%%% - Preamble
\documentclass[11pt]{article}
\usepackage[sans, stdmargin, noindent]{rajeev}
\usepackage{lastpage}
\usepackage[inline]{asymptote}
\usepackage[normalem]{ulem}
\pagestyle{fancy}
\rhead{Last Updated: \today}
\lhead{$\RR^n$ Bonus Problems}

\begin{document}

\begin{center}
    \Large \textbf{$\RR^n$ Bonus Problem \#3}
\end{center}
\begin{center}
    \Large Rajeev Atla
\end{center}

\section{Problem}
\sout{Settlers of Catan}
A board game is played on a hexagonal grid of 19 tiles.
A 'traveler' token starts on the center tile.
Each turn a die is rolled to determine what neighboring tile the traveler moves to (all six directions equally likely).
The turn that the traveler leaves the board, the game ends.
What is the expected number of turns of the game?

\section{Diagram}
\begin{center}
    \begin{asy}
        size(10cm);

        pair[] coords = {
            (0, 0),
            (0, sqrt(3)), (-3/2, sqrt(3)/2), (-3/2, -1*sqrt(3)/2), (0, -sqrt(3)), (3/2, -1*sqrt(3)/2), (3/2, sqrt(3)/2),
            (3/2, 3*sqrt(3)/2), (0, 2*sqrt(3)), (-3/2, 3*sqrt(3)/2), (-3, sqrt(3)), (-3, 0), (-3, -sqrt(3)), (-3/2, -3*sqrt(3)/2), (0, -2*sqrt(3)), (3/2, -3*sqrt(3)/2), (3, -sqrt(3)), (3, 0), (3, sqrt(3))
            };
        

        for(int n = 0; n < coords.length; ++n){
            draw(shift(coords[n])*polygon(6), blue);
            dot(coords[n], red);
            label(string(n), (coords[n]), S);
        }

        pair[] coords2 = {
            (3, 2*sqrt(3)), (3/2, 5*sqrt(3)/2), (0, 3*sqrt(3)),
            (-3/2, 5*sqrt(3)/2), (-3, 2*sqrt(3)), (-9/2, 3*sqrt(3)/2),
            (-9/2, sqrt(3)/2), (-9/2, -sqrt(3)/2), (-9/2, -3*sqrt(3)/2),
            (-3, -2*sqrt(3)), (-3/2, -5*sqrt(3)/2), (0, -3*sqrt(3)),  
            (3/2, -5*sqrt(3)/2), (3, -2*sqrt(3)), (9/2, -3*sqrt(3)/2), 
            (9/2, -sqrt(3)/2), (9/2, sqrt(3)/2), (9/2, 3*sqrt(3)/2)
            };

        for(int n = 0; n < coords2.length; ++n){
           dot(coords2[n], green);
           label(string(n+19), (coords2[n]), S);
       }

    \end{asy}
\end{center}

\section{Solution}
We wish to find the expected value of the number of turns in the game, which we denote $N$.

\[
    \E{N} = \sum N\ \P{N}
\]

The dice is truly random, so there is no upper bound on $N$.
We note that this game is really akin to a Markov chain, in that it doesn't matter what the past states are.


\end{document}
